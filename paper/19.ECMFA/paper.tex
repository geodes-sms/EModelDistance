\documentclass{jot}

\usepackage[utf8]{inputenc}
%\usepackage[T1]{fontenc}
\usepackage[english]{babel}
\usepackage{microtype} % optional, for aesthetics
\usepackage{tabularx} % nice to have
\usepackage{booktabs} % necessary for style

%\usepackage[bookmarks,bookmarksopen,bookmarksdepth=2]{hyperref} %sections as bookmarks in adobe pdf
%\hypersetup{pdfpagemode=UseNone}


%%% Article metadata
\title{Domain-Specific Model Distance Measures}
\runningtitle{Domain-specific distance measures}

\author[affiliation=UL, nowrap] % , photo=FILE]
    {Manuel Wimmer}
    {is a ... in ... at ....
    Contact him at \email{EMAIL}, or visit \url{URL}.}

\author[affiliation=UdeM, nowrap] % , photo=FILE]
{Eugene Syriani}
{is an associate professor in the department of computer science and operations research at Universit{\'e} de Montr{\'e}al.
	Contact him at \email{syriani@iro.umontreal.ca}, or visit \url{www.iro.umontreal.ca/~syriani}.}

\author[affiliation=UV, nowrap] % , photo=FILE]
{Robert Bill}
{is a ... in ... at ....
	Contact him at \email{EMAIL}, or visit \url{URL}.}

\affiliation{UdeM}{Universit{\'e} de Montr{\'e}al}
\affiliation{UV}{University of Vienna}
\affiliation{UL}{University of Linz}

\runningauthor{Wimmer et al.}

\jotdetails{
    volume=V,
    number=N,
    articleno=M,
    year=2011,
    doisuffix=jot.201Y.VV.N.aN,
    license=ccbynd % choose from ccby, ccbynd, ccbyncnd
}

% Put edit comments in a really ugly standout display
\usepackage{ifthen}
\usepackage{amssymb}
\newboolean{showcomments}
\setboolean{showcomments}{true} % toggle to show or hide comments
\ifthenelse{\boolean{showcomments}}
{\newcommand{\nb}[2]{
		\fcolorbox{gray}{yellow}{\bfseries\sffamily\scriptsize#1}
		{$\blacktriangleright$#2$\blacktriangleleft$}
	}
	\newcommand{\version}{\emph{\scriptsize$-$working$-$}}
}
{\newcommand{\nb}[2]{}
	\newcommand{\version}{}
}

\newcommand\es[1]{\nb{ES}{\textcolor{red}{\textsl{#1}}}}
\newcommand\mw[1]{\nb{MW}{\textcolor{red}{\textsl{#1}}}}
\newcommand\rb[1]{\nb{RB}{\textcolor{red}{\textsl{#1}}}}

% For URLs in the references
\usepackage{url}
\usepackage{hyperref}
%\usepackage{verbatim}

% Advanced Math
%\usepackage{amsmath}
%\usepackage{amssymb}	% double line font letters (for number sets N,Z,D,Q,R)

% Images and Floats
\usepackage{epsfig}
%\usepackage{epstopdf}

%\usepackage{caption}
%\usepackage{subcaption}
%\usepackage{lscape}			% landscape
%\usepackage{array}			% align vertically in tables
%\usepackage{multirow}
%\usepackage{rotating}

% Theorems, Definitions, ... (always after 'amsmath')
%\usepackage{amsthm}
%\theoremstyle{definition}
%\newtheorem{prop}{Property}%[section]
%\newtheorem{defn}{Definition}%[section]

% Algorithm environment
%\usepackage{algorithm}
%\usepackage{algorithmic}

% Commutative diagrams
%\usepackage{diagrams}

%\usepackage{enumitem}

%\usepackage{listings}

%%%%%%%%%%%%%%%%%%%%%%%%%%%%%%%%%%%%%%%%%%%%%

% Standard shortcuts
\newcommand{\eg}{e.g.,~}										% exempli gratia (for the sake of example)
\newcommand{\ie}{i.e.,~}										% id est (that is)
\newcommand{\etal}{~et al.}									% et alia (and others)
\newcommand{\Fig}[1]{Figure~\ref{#1}}  			% choose Fig. or Figure, depending on the style
\newcommand{\Table}[1]{Table~\ref{#1}}	    % Table reference
\newcommand{\Sect}[1]{Section~\ref{#1}}	  	% section name always with a capital S
\newcommand{\Model}[1]{\textsf{\small{#1}}} % name of any modeling artifact (e.g., formalism, model element, rule, ...)
\newcommand{\Code}[1]{\texttt{\small{#1}}}	% inline code
\providecommand{\e}[1]{\ensuremath{\times 10^{#1}}}	% scientific notation: x.10^y

%%%%%%%%%%%%%%%%%%%%%%%%%%%%%%%%%%%%%%%%%%%%%


\begin{document}

\begin{abstract}
A lot of research was invested in the last decade to develop differencing methods for models to identify the changes performed between two model versions. 
A difference model captures these changes. However, less attention was paid to distance computations of model versions. While different versions of a model may have a similar amount of differences,
the distance to a base model may be drastically different. Therefore, we present in this paper distance metrics for models, a method to automatically generate tool support for computing distance measures and show the benefits of distance measures over differences in searching for model evolution explanations. The results of running different experiments show...
\end{abstract}

\keywords{Three; Keywords.}

\section{INTRODUCTION}
With the emergence of model-driven engineering~\cite{}, the need for dedicated techniques for management models arose. In particular, a lot of research was invested in the last decade to develop differencing methods for models to identify the changes performed between two model versions. Different algorithms for computing differences as well as several representations of differences have been proposed. Difference models which capture the changes between model versions have been used in different cases such as model co-evolution, versioning or synchronization. 
Thus, difference models represent an important artefact kind in model-driven engineering. 

While differences have been in the focus, less attention was paid to distance computations of model versions. Distances may be useful because of several reasons. First, while different versions of a model may have a similar or even same amount of differences, the distance to a base model may be drastically different. Second, distances may be an additional metric to consider the evolution paths of models to reach a certain setting. As an example consider the movement of attributes between different classes. If we have classes A, B, C, D all connected in sequence with an association, and move an attribute from A to another class, we may always have the same difference: attribute deleted in A and added to one of the other classes. However, a distance metric could tell us how far we have moved the attribute by getting a different distance measure if the attribute ends up in class B, C, or D.  

In order to provide additional measures for understanding how much a model has changed, we present in this paper distance metrics for models. In addition, we present a method to automatically generate tool support for computing domain-specific distance measures from an annotated version of the metamodel of the language used for defining the models. Finally, we show the benefits of using distance measures over differences in searching for model evolution explanations. Specifically, the results of running different experiments show...

This paper is organized as follows. In Section 2, we provide an overview on the background of our approach and discuss the present gap between differences and distances. Our approach for generating tool support for computing domain-specific distance measures from metamodels is presented in Section 3. In Section 4, we evaluate the effectiveness of domain-specific distance measures for the particular case of computing model evolution explanations in terms of domain-specific change operations. In Section 5, we survey related work and in Section 6, we conclude with a short summary and an outlook on future work. 

\section{BACKGROUND}
In this section, we explain the background of our work. As every software artifact, also software models are subject to continuous evolution. Knowing the operations applied between two successive versions of a model is not only crucial for helping developers to efficiently understand the model's evolution (Koegel et al., 2010), but it is also a major prerequisite for model management tasks REF. In general, we may distinguish between two categories of model diffing approaches. The first category describes model diffs as atomic operations, such as additions, deletions, updates, and moves of model elements. The second category uses domain-specific operations (Sunyé et al., 2001) consisting of a set of cohesive atomic operations, which are applied within one transaction to achieve one common goal often expressed as a kind of model transformation. We detail both categories in the following.

\subsection{Model Diffs as Atomic Operations}

Current model comparison tools often apply a two-phase process: $(i)$ correspondences between model elements are computed by model matching algorithms (Kolovos et al., 2009), and $(ii)$ a model diffing phase computes the differences between two models from the established correspondences. For instance, EMF Compare (Brun and Pierantonio, 2008) – a prominent representative of model comparison tools in the Eclipse ecosystem – is capable of detecting the following types of atomic operations:

\begin{itemize}
  \item Add: A model element only exists in the revised version.
  \item Delete: A model element only exists in the origin version.
  \item Update: A feature of a model element has a different value in the revised version than in the origin version. This includes both attribute values and references.
  \item Move: A model element has a different container in the revised version than in the origin version.
\end{itemize}

The advantage of diffing for atomic operations is that generic tool support is available which works for all types of models. On the downside, working with large atomic differences is challenging and may require a higher level of abstraction. Therefore, the following diffing approaches have been developed over the last years. 


\subsection{Model Diffs as Domain-Specific Operations}

To raise the level of abstraction for model diffs, a more concise view of model differences is required that aggregates the atomic operations into domain-specific operation applications such that the intent of the change is becoming explicit. Existing solutions (Hartung et al., 2010, Küster et al., 2008, Xing and Stroulia, 2006, Langer, Kehrer) provide language-specific operation detection algorithms. Often model transformations are the technique of choice used for specifying executable domain-specific operations. In particular, the operations are specified by transformation rules stating the operation's preconditions, postconditions, and actions that have to be executed for applying the operation. Especially, the approaches  proposed by Langer et al. and Kehrer et al. build on model transformations for operation detection between two model versions. The output of these approaches is a set of transformation rule applications which correspond to the list of domain-specific operation applications. By following these approaches, the difference models can be compressed as shown in different studies REF. However, finding the set of domain-specific operations which best describes a model evolution is a challenging problem, since there are many different evolution paths between two versions and there are dependencies between the execution of the operations which may mask some of them in the revised version of a model. Therefore, search-based approaches have been developed which evaluate different evolution path with respect to the ability if they can produce the revised version of model. Again, for this atomic differences have been used to compare the computed revised versions with the given revised versions. 

\subsection{Synopsis}

While atomic operations allow to reason about the differences between two models on a fine-grained level, they fail to reason on distances. 
Assume just the simple Pac-Man example presented in Figure X. The Pac-Man is moving from one area to a very different one in the game field. The same difference is reported for all the intermediate version, namely a change in the assignment of the Pac-Man to the fields. This means, if the Pac-Man just moves to the neighbour field or to the completely other end of the game field, the difference is the same as atomic diffing techniques are fully agnostic about the spatial aspects of elements. Domain-specific operation detection approaches would be more suitable to represent distances as they are able to represent differences as a set of operations. However, for their computation, atomic diffing is required which is not providing an appropriate guidance to decide if the search goes into the right direction or not. For instance, the Pac-Man could move to any other field and we always get the same fitness value of operations sequences. Therefore, we propose the usage of additional distance metrics to provide additional information on top of difference models in the following section and show their benefits for the search of domain-specific operation-based differences in Section 4.   

\section{DOMAIN-SPECIFIC DISTANCES}
\subsection{Running example}

\begin{figure}
    \centering
    \includegraphics[width=\linewidth]{images/pacman_example}
    \caption{The initial and target models of the Pacman game configuration}
    \label{fig:example}
\end{figure}
%
We rely on the running example of a simplified Pacman game, a well-known game where Pacman navigates through grid nodes searching for food to eat, while ghosts try to kill him.
We implemented a DSL to define game configurations, based on~\cite{Syriani2013a}.
\Fig{fig:example} illustrates two Pacman game models in the concrete syntax of the DSL.
Pacman, food, and ghosts are positioned grid nodes with an \texttt{on} reference.
Grid nodes are connected by \texttt{left}, \texttt{right}, \texttt{up}, and \texttt{down} references to define the permissible navigation of Pacman and ghosts.
A score object keeps track of the number of food Pacman eats.
We define the operational semantics of the DSL in terms of an inplace model transformation, implemented with graph transformation rules as in~\cite{Syriani2013a}.
One rule represents Pacman eating food on a grid node and updating the score.
Another represents the ghost killing Pacman when they are on the same grid node.
Four rules for Pacman and four others for the ghost represent moving in each direction to an adjacent grid node.\es{should we show one rule in Henshin?}
Although the rules should obey a certain scheduling, \eg killing has priority over moving to end the game, in this work we assume a graph grammar \ie any rule can be applied at any time during the execution of the transformation\es{do we need this assumption?}.

Our goal is to find the minimal sequence of rule applications starting from the initial model leading to the target model.
Search-based techniques are very useful to solve this problem.
They explore all possibilities of the search-space by generating intermediate models as a result of applying the rules, while optimizing the objective to get closer to the target model.
For the example in \Fig{fig:example}, a minimal rule sequence is Pacman moves up once, then moves right three times, each time eating the food.
The ghost also moves left twice.
\es{In current approaches? cite?}To compute the difference between the two models, a generic model comparison tool, like EMFCompare, would report that three food objects are deleted, two \texttt{on} references (Pacman and ghost) are changed, and the attribute value of the score is modified.
This information is not precise enough to identify the minimal sequence of rules.
For example, if Pacman moves right first, then he will have to go through the top center grid node at least twice.
Common model difference approaches rely solely on changes in the abstract syntax.
However, the comparison needs to be tailored to both the DSL and its semantics to find the best rule sequence.
In this example, the notion of Pacman and ghost movements must be encoded in the comparison.
Inplace model transformation rules typically encode semantical operations, such as operational semantics or refactoring~\cite{Lucio2016}.
Therefore, we propose a set of domain-specific distance metrics that take into consideration abstract syntax changes as well as semantics of the transformation to optimize the search space for identifying the minimal sequence of rule applications.

\es{to be used where needed}\begin{itemize}
    \item[Regular difference results:]
        3 food objects deleted ;
        2 on references modified ;
        1 score value attribute modified.
    
    \item[Distance results:]
        Move distance is $3+2=5$ ;
        Value distance is $\frac{|4-1|}{4}=0.75$ ;
        Element distance is $\frac{3+0}{13+10}=0.13$.
    
    \item[Minimal rules applied:]
        1 Pacman Move Up ;
        2 Pacman Move Right ;
        3 Pacman Eats Food ;
        2 Ghost Move Left.
\end{itemize}

\subsection{Model distance metrics}

The idea is to generate the optimizing search problem to be tailored for the domain. In this sense, we customize the fitness function based on a distance metric. The metrics to minimize are:
\begin{itemize}
	\item[Move distance:] There is often movement of elements (\eg pacman moving on the grid, attributes moving around classes). The move distance of a movable object is the length of the shortest path from its position in M1 to its position in M2. Move distance is related to computing the difference with Ecore references.
	\item[Element distance:] Is the difference in the presence/absence of elements in M1 and M2. The element distance is the ratio, between 0 and 1, of the number of differences with respect to the total number of objects in M1 and M2.
	\item[Value distance:] Is the difference in attribute values between M1 and M2. We assume that any attribute type can be encoded as numbers. Then the value distance of attribute x is its margin of error: $|M2.x - M1.x| / M2.x$. In this case, M2 acts as the expected target.
	\item[Rule application distance:] Is the number of rules to apply to get from M1 to M2. This distance is already taken into account in MOMot and returns a positive integer.
\end{itemize}

The \emph{Move distance} relies on the Floyd-Marshall algorithm that computes the shortest from any position element to any position element in an Ecore model. This code is in EcoreShortestPaths and is independent from the domain. Its only external dependency is an interface IEReferenceNavigator which provides a function that gives the neighbor(s) of a position element. DistanceUtil provides all the necessary methods the move distance requires.

The \emph{Value distance} is an average of all attribute distances. We only consider attributes of objects present in both M1 and M2 because the element distance takes care of absence and presence of elements. We assume that any attribute value can be represented as a unique number. DistanceUtil provides the toDouble method to perform that conversion. Currently, it only supports number values encoded as Number or String data types. For each attribute, we compute its margin of error.

The \emph{Element distance} looks for the objects present in M1 but not M2 and M2 but not M1. This is then divided by the size of M1 and M2. We only consider objects instances of metamodel classes. So references and attributes are not taken into account in this measure. This distance relies on the unique ID of each object as returned by the getId method.

DistanceCalculator is the abstract class at the root that should be inherited by your distance function. For example MoveDistance only relies on the move distance between M1 and M2.

\subsection{Generation of domain-specific metrics}

This implementation is just a proof of concept. It has been implemented with the mindset that the distance calculation is generated automatically from analyzing the metamodel and the transformation rules.

Given a metamodel MM and Henshin rules R, we want to generate the distance calculator that will be used by the MOMot script.

The domain-specific distance classes (e.g., the move, element, value distances) can be easily generated. They have two dependencies to the metamodel:

The package instance, by overriding the getEPackageInstance() function
The constructor, by instantiating the appropriate DistanceUtil singleton object specific to the metamodel

Only the utility class (e.g., PacmanDistanceUtil) must be generated after analyzing MM and R. Following the code in PacmanDistanceUtil should guide you to know how to generate the code. Here are special considerations:

Your Utility class must inherit from DistanceUtil.
It should import all movable, position, modifiable, and other classes.
It must provide a function used for its singleton instantiation as follows.

It should have 4 attributes for movable, position, modifiable, and all other types.

It should override all abstract methods from DistanceUtil.
The tricky part is the generation of the Object getId(EObject object) method. If each object has attribute with setID(true) you can rely on super. Otherwise, you have to make up one unique on your own. For example, if you know for sure that an attribute is unique, then you can rely on it (e.g., Places and Transitions in the Petrinet example). If there is only one possible instance of this element, then return true (e.g., Scoreboard in the Pacman example). It can also rely on an object it references or that references it (e.g., Food in the Pacman example).

You also need to generate the DistanceUtilFactory specific to the metamodel (e.g., PacmanGameDistanceUtilFactory). Its only purpose is to make the concrete DistanceUtil accessible as a singleton.

\subsubsection{Customization}

The only code that is specific to the metamodel is the class that implements DistanceUtil. This is where you define the functions that provide the following information:

\begin{itemize}
	\item[The movable objects:] An object is movable if, when analyzing the rules, it has a reference to a position object and the rules modify that reference. Note that it could also be that a position object references a movable object.
	\item[The position objects:] An object is a position if, when analyzing the rules, it is what movable objects are always linked to. Note that it could also be that a position object references a movable object.
	\item[The modifiable objects:] An object is modifiable if, when analyzing the rules, one of its attributes changes value.
	\item[The other objects:] Any object that is not movable, position or modifiable.
	\item[The ID of an object:] the value that uniquely identifies an object. This is used to find similar elements between M1 and M2.
	\item[The modifiable attributes:] All attribute values subject to modification for a given object.
	\item[Accessing the position:] the attribute used to know the position of a movable object.
	\item[Accessing the neighbors of a position:] the attribute used to connect position objects.
	\item[Accessing the root:] used to find the root object of M1 and M2.
\end{itemize}

\section{EVALUATION AND DISCUSSION}
In this section, we present experiments to understand if the distance metrics introduced before help in model management tasks in comparison to diff models. In particular, we focus on the use case presented in the previous section, namely operation-based change model recovery. 

All experiments use the same input and output models, the same transformations, the same solution length and the
same optimization algorithms and only differ in the fitness functions used in these algorithms, i.e., distance vs. difference metrics. Two cases, namely the Pacman and Petri net cases, have been freshly created for this evaluation, but are commonly used as examples in different studies. The Refactoring case has been used to evaluate different algorithms for MoMOT in the past, where all algorithms optimized the same fitness function. Here, this case is reused to evaluate the same algorithm with different fitness functions.

\subsection{Objectives}

As our main goal is to compare different fitness functions in terms of their impact on search processes, we want to answer two research questions:

\begin{itemize}
	\item \textit{RQ1 - Search Space Exploration}: Is there a significant difference in the solutions found by applying each fitness function?
	\item \textit{RQ2 - Search Time}: Is there a significant difference in the number of iterations required to get good solution when using two different fitness functions?
	%Senseless - you can easily define simple generic model distance functions which are much faster, e.g. by just mapping things to feature vectors
	%EMFCompare might be really slow, but that isn't everything
	%\item \textit{RQ3 - Search Time}: Is there a significant difference in the time
\end{itemize}

We answer these research questions by measuring several properties of the final and intermediate results during each experimental run. In particular, to answer RQ1, we compare the final solutions, while we compare also the intermediate results to answer RQ2.

\subsection{Experiment setup}

\subsubsection{Selected cases}

We evaluate these research questions with three MoMOT case studies. Each case study consists of a single domain model and multiple Henshin transformations.
The \textit{Pacman} case study has been introduced in Section~\ref{sec:motivating}. The \textit{Petrinet} case study simulates a Petrinet with multiple tokens which can be in different places. A token can be transferred to another place by firing transitions. The \textit{Refactoring} case study, as found in \cite{?}, is about performing refactoring operations like extracting superclasses or pushing up attributes to a model containing classes and attributes.
For each case, we randomly generated multiple test input and output models as outlined in following paragraphs.

\begin{table}
\centering
\begin{tabular}{|c|c|c|c|}
\hline
 & \textbf{Pacman} & \textbf{Petrinet} & \textbf{Refactoring} \\
\hline
\textbf{Rule count} & High & Low & Medium \\
\hline
\textbf{Rule complexity} & Low & Low & High \\
\hline
\textbf{Solution length} & High & High & Low \\
\hline
\textbf{Structural changes} & No distance changes & - & Distance changes \\
\hline
\end{tabular}
\end{table}

The case studies have been selected as they differ in terms of rule count, rule complexity, and expected solution length as detailed in Table~\ref{tab:casecomp}.
The Petri net simulation contains only a single rule which can match in many different ways and whose application does not limit its re-execution. Thus, the rule count is low, but the expected solution length is high. In contrast, the rules for refactoring case typically can be applied in a more limited way and applying it typically limits the application even more, yielding a lower expected solution length. The Pacman example is somewhere in between as many rules are defined, but they semantically all do similar things, which is moving Pacman or a Ghost, but only differ in what happens on specific fields.
Most importantly, only the refactoring case contains rules which modify the graph in a way which matters for the domain specific distance evaluation. A detailed description of the cases follows in the next paragraphs.

\paragraph{Pacman} The pacman example was described in the previous sections and is thus not explained further here.

\paragraph{Petri nets}
This example consists of places which may be connected to other places via transitions. A place can have multiple transitions and each transition can have multiple outgoing states, contributing to a non-determinism in firing the rules. While places and transitions are modeled as named objects, tokens are not objects, but are just modeled as attribute values.

\begin{figure}
\centering
\includegraphics[width=0.8\textwidth]{images/firetransition.png}
\caption{A transition can fire if there is at least a single token in each source place}
\label{fig:firerule}
\end{figure}

There is only a single rule \textit{fireRule} as shown in Fig.~\ref{fig:firerule} which allows to move a single token from a place to another place. \mw{you may have several incoming arcs to a transition, i.e., several incoming places}
As required for MoMOT, the parameters of the rule uniquely identify the rule match. Even though the model is simple, the number of models generated by applying this rule multiple times can be huge, especially if multiple tokens are used.

\paragraph{Object-oriented refactoring}

This example~\footnote{Also see \url{http://martin-fleck.github.io/momot/casestudy/class_restructuring/} for a detailed description}, which was initially taken from the TTC 2013, contains entities with named, typed properties and generalizations.
Originally, it was used as optimization problem to reduce the number of attributes and entities in a system. However, in this paper
we want to find a way to refactor a system in a particular fashion. The three refactoring rules are (a) Pull up attribute, which moves an attribute existing in all subclasses into the superclass, (b) extract super class, which creates a new superclass if an attribute is existing in some, but not all subclasses of a particular class, or (c) create root class, which is the same if the classes do not have any existing superclass.

In contrast to both examples above, this example changes the model structure by adding entities and generalizations and removing properties.

\subsubsection{Analysis Procedure}

While many general difference metrics exists, we opted for the ones provided by EMF Compare as this tool is widely used for diffing models. In particular, we calculate the difference between two models by counting the percentage of model differences in the whole model.

Initially, we generated 100 examples of pairs of source and target model for each case. As the results for the refactoring case initially were good, but not statistically significant, we generated 1000 examples for this case as well. While the source model was generated randomly within
certain predefined parameters, the target model was generated by randomly applying transformations to the source model. However, for the Pacman case, we had to add more ``intelligent'' rules to generate sensible target states as typically, randomly applying move operations just lead to Pacman being killed soon by running into a ghost or just being eaten by a ghost. Here, the rules force Pacman to eat neighboring food when possible and avoiding any ghosts. The search process itself does not use these rules, but just the generic ones.

For each example, only a single run was conducted. Then, the final solutions generated by each run were compared. A solution is considered better if the number of transformations used to reach the target model is smaller. If a run did not produce any transformation sequences matching the target model, this number is considered to be infinite. We did not consider cases where the result models were not exactly matching the target model as at least two different distance metrics could be used for that.

To determine which fitness function was better we assumed that if both fitness functions would produce equal results, there was a 50/50 chance that either fitness function was better for runs which produced differences.
For each example, we calculated the chance that the distance metrics would be as good or better just by chance.
Thus, we can reject the null hypothesis that the domain specific fitness function is not better if this value is below 5\%.

\mw{yes we should describe here the }

%Values are so low - calculate p exactly!

\subsubsection{Results}

In the following, we describe the results of the experiments conducted in terms of the research questions.

\subsubsection{Results for RQ1 Search Space Exploration}

\begin{table}
\centering
\begin{tabular}{|c|c|c|c|c|}
\hline
 & Generic & Equal & Domain-specific & p-value \\
\hline
Pacman & 1 & 73 & 27 & \textbf{.11E-7} \\
\hline
Petrinet & 26 & 30 & 45 & \textbf{.016} \\
\hline
Refactor (100) & 22 & 47 & 32 & .11 \\
\hline
Refactor (1000) & 243 & 488 & 270 & .13 \\
\hline
\end{tabular}
\label{tab:resultsrq1}
\end{table}

Table~\ref{tab:resultsrq1} shows the differences in the results quality for both approaches. 
\mw{we should move the following part of the paragraph to the setup: selected case} For the Pacman case, we generated 8x8 fields with one Pacman, three ghosts, and 15 food, where all food and all other entities were randomly distributed. The Petrinet case used 10 places and 1 to 2 transitions per place and 1 to 2 outgoing place per transition. Also, the net was initialized with between 2 and 4 tokens. The Refactor case used 6 entities, 1 to 5 attributes per entity, where each name was one of 8 possible names and each attribute name was associated to 1 to 3 types. Also, each entity had a 50\% chance to have a random superclass.

The Pacman case was rather hard for both fitness functions - all the equal values are due to no solution being found in either case. However, in this instance the domain specific fitness function could show its benefits, as it could solve the problem in 27 cases, while the generic one could only solve it in one case, yielding a significant difference. In contrast, the Petrinet case was rather easy for both algorithms, as most of the equal values were due to both algorithms finding solutions having the exact same quality. Still, the domain specific fitness function allowed us was at least statistically significantly better. In contrast, we could not achieve statistically significantly better results in the refactor case. While there was a trend towards the domain specific distance function, this trend did not turn into a statistically significant advantage for larger example sizes.

\mw{we need a final answer to RQ1}

\subsubsection{Results for RQ2 Search Time}

We determined the rate of convergence by storing the solutions found after each evaluation step



%* Linearize quality to make comparable
% - Alternative: Just make "`normal"' comparison in each step?

%* Calculate average solution quality over the course of the runs
% - Save solutions after each step
% - Maybe different distance function implement? Normalize with initial difference
% - Just show it, values ... I am not sure that they would be sensible here

While exact time measurements were not taken, the domain-specific distance function always was faster, sometimes drastically faster,
than the generic EMF-compare based one.

%- Time: EMFCompare takes longer (difficult to measure reliable, as performance VS solution quality metrics collide)
%- Unexpected: EMFCompare converges faster initially for many models - why?
% * Distance metrics used is EMFCompare ...

\subsection{Threats to validity}

\subsection{Discussion}
The current implementation assumes that M1 and M2 have the same position objects (none are deleted or created). This is too restrictive. Therefore M1 and M2 should be preprocessed by merging their position elements and then use the move distance. One possibility is to use EMFCompare to merge M1 and M2 based on the position elements.

Interestingly, when you run the Pacman test with models/input12missing.xmi and models/targetNoPac.xmi in one signle run, you don't necessarily get always the same sequence of rule applications. That is because p1 can be killed anywhere along the path of g1. 

\section{RELATED WORK}
With respect to the contribution of this paper, we discuss two threads of related work. First, we discuss approaches for model differencing. Second, we discuss approaches which cluster models based on distance metrics in the context of analysing model repositories.

\es{The next 3 paragraphs are here in case we want to reuse them}

This section reviews related works on model differencing (see also~\cite{StephanC13} for a survey on
model comparison approaches and applications)

Kolovos et al.~\cite{Kolovos2009} survey current approaches for model matching. These can be: {\em static identity-based}, 
which assume a unique identifier for objects; {\em signature-based}, which compare objects based on a dynamic signature 
calculated from the objects' properties; {\em similarity-based}, which match objects based on the aggregated 
weighted similarity of their properties, but obviates the model semantics; and {\em language-specific}, developed 
ad-hoc for a modeling language and its semantics. For example, using {\em signifiers}~\cite{LangerWGKV12} 
(\ie combinations of features of a metamodel class) as comparison criteria falls in the signature-based category, 
EMFCompare is similarity-based but permits defining custom matching algorithms, and UMLDiff is language-specific. 
In general, each solution is better fit for certain kinds of problems: a language-specific matching algorithm may 
be faster and more accurate than a generic algorithm, but its implementation requires more effort.

Maoz et al.~\cite{MaozRR10} argue that existing model differencing approaches are purely syntactic and 
challenge the community to develop semantic diff operators. These calculate a set of diff witnesses that give 
a proof of the real change between two models and the effect on their semantics. Two models may be syntactically 
different but have no diff witnesses, meaning that they are semantically equivalent. For example, a diff witness 
of two class diagrams would be an object diagram that is an instance of one of the class diagrams but not of 
the other, while for activity diagrams, it would be an execution trace admitted by only one of the diagrams. 
Diff witnesses also allow deciding whether the semantics of two versions of a model are equivalent, incomparable, 
or one refines the other. This approach was later realized in the Diffuse framework~\cite{MaozR18}.
Extending our approach to deal with model diffs concerned with the instantiability or 
executability of models as comparison criteria is left for future work.

\subsection{Model Differencing} In the last decade, there have been many works published for deriving model differences which are based on atomic operations, e.g., see~\cite{} for concrete approaches and~\cite{} for surveys. In~\cite{} an approach is presented to derive a domain-specific difference language for atomic changes for a particular metamodel.
 To the best of our knowledge, these approaches are not supporting the computation of distances as we propose in this paper.
In addition to these approaches focusing on atomic operations, there have been also some works published on the detection of domain-specific operations. For instance, Xing and Stroulia [10] present an approach for detecting refactorings in evolving software models which is integrated in UMLDiff. Refactorings
are expressed by change pattern queries used to search a
difference model obtained by a state-based model comparison.
The approach by Vermolen et al. [11] copes with the detection
of complex evolution steps between different versions of a
metamodel to allow for a higher automation in model
migration. They use a diff model comprising primitive changes
as input and calculate, on this basis, complex changes. The
approach is tailored to the core of object-oriented
metamodeling languages, but follows a similar methodology as
UMLDiff. However, a specific feature is the detection of so-called
masked changes, i.e., changes, which are hidden by other
changes in a way that their effect is partially or also totally
missing in the revised model, by defining additional detection
rules. Furthermore, there is the work of Küster et al. [12] for
calculating hierarchical change logs including compound
changes in the absence of recorded change logs. The authors
apply the concept of Single-Entry-Single-Exit fragments to
calculate the hierarchical change logs after computing the
correspondences between two process models. Thereby,
several atomic changes are hidden behind one compound
change. The use of graph transformations to collect atomic
changes on models into more meaningful changes called user-level
changes has been reported in [13].

In~\cite{} we have presented an domain-specific operation detection approach which transforms model transformation rules into diff patterns which can be matched on diff models. In follow-up work, we presented the first search-based approach for detection operation sequences between two model versions without requiring a diff model as basis to detect operations. However, we resorted on atomic diff models in the fitness function to compute how close a computed model is with respect to the given revised model. 

To sum up, many approaches have been proposed to compute difference models in the past. However, to the best of our knowledge, none focussed on distances as we present in this paper. Also the comparison of using differences or distances for searching for operation sequences is a novel contribution of this paper. 

\subsection{Model Clustering}

Recent work discusses dedicated support to cluster models in model repositories~\cite{}. Thus the setting is different to the two-way or three-way comparison of models which is mostly studied in the model comparison and model versioning fields. In such cases, it is assumed that many models have to be compared at once. Thus, a generic clustering technique for models is proposed which is based on the translation of model to a vector space model. Based on this vector space representation, clustering distance measures can be reused such as Manhattan distance.

\url{http://ceur-ws.org/Vol-2245/ammore_paper\_2.pdf}

\url{http://www.scitepress.org/DigitalLibrary/Link.aspx?doi=10.5220\%2f0005799103610367}



\section{CONCLUSION}
In this paper we have introduced a novel notion of metrics for model comparison, so-called distance metrics. Furthermore, we defined a method to derive measurement tool support from the language definitions of modeling languages and have shown their usage in the recovery of operation-based change logs between two model versions. 

Heavily changing model structures seems to be a particular challenge to compute distance metrics. The used cases in our experiment have shown different performances of calculating and using model distances. As future work, we plan to perform more experiment with the refactoring cases where elements are deleted, created, and potentially re-created. Finally, we plan to use model distances for other model management tasks such as model synchronization where currently also often model differences are used to guide the synchronization process. 

\bibliographystyle{alphaurl}
\bibliography{biblio}

\abouttheauthors

\begin{acknowledgments}
This work has been sponsored by
\end{acknowledgments}

\end{document}
