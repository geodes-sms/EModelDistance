In this section, we explain the background of our work. As every software artifact, also software models are subject to continuous evolution. Knowing the operations applied between two successive versions of a model is not only crucial for helping developers to efficiently understand the model's evolution (Koegel et al., 2010), but it is also a major prerequisite for model management tasks REF. In general, we may distinguish between two categories of model diffing approaches. The first category describes model diffs as atomic operations, such as additions, deletions, updates, and moves of model elements. The second category uses domain-specific operations (Sunyé et al., 2001) consisting of a set of cohesive atomic operations, which are applied within one transaction to achieve one common goal often expressed as a kind of model transformation. We detail both categories in the following.

\subsection{Model Diffs as Atomic Operations}

Current model comparison tools often apply a two-phase process: $(i)$ correspondences between model elements are computed by model matching algorithms (Kolovos et al., 2009), and $(ii)$ a model diffing phase computes the differences between two models from the established correspondences. For instance, EMF Compare (Brun and Pierantonio, 2008) – a prominent representative of model comparison tools in the Eclipse ecosystem – is capable of detecting the following types of atomic operations:

\begin{itemize}
  \item Add: A model element only exists in the revised version.
  \item Delete: A model element only exists in the origin version.
  \item Update: A feature of a model element has a different value in the revised version than in the origin version.
  \item Move: A model element has a different container in the revised version than in the origin version.
\end{itemize}

The advantage of diffing for atomic operations is that generic tool support is available which works for all types of models. On the downside, working with large atomic differences is challenging and may require a higher level of abstraction. Therefore, the following diffing approaches have been developed over the last years. 


\subsection{Model Diffs as Domain-Specific Operations}

To raise the level of abstraction for model diffs, a more concise view of model differences is required that aggregates the atomic operations into domain-specific operation applications such that the intent of the change is becoming explicit. Existing solutions (Hartung et al., 2010, Küster et al., 2008, Xing and Stroulia, 2006, Langer, Kehrer) provide language-specific operation detection algorithms. Often model transformations are the technique of choice used for specifying executable domain-specific operations. In particular, the operations are specified by transformation rules stating the operation's preconditions, postconditions, and actions that have to be executed for applying the operation. Especially, the approaches  proposed by Langer et al. and Kehrer et al. build on model transformations for operation detection between two model versions. The output of these approaches is a set of transformation rule applications which correspond to the list of domain-specific operation applications. By following these approaches, the difference models can be compressed as shown in different studies REF. However, finding the set of domain-specific operations which best describes a model evolution is a challenging problem, since there are many different evolution paths between two versions and there are dependencies between the execution of the operations which may mask some of them in the revised version of a model. Therefore, search-based approaches have been developed which evaluate different evolution path with respect to the ability if they can produce the revised version of model. Again, for this atomic differences have been used to compare the computed revised versions with the given revised versions. 

\subsection{Synopsis}

While atomic operations allow to reason about the differences between two models on a fine-grained level, they fail to reason on distances. 
Assume just the simple Pac-Man example presented in Figure X. The Pac-Man is moving from one area to a very different one in the game field. The same difference is reported for all the intermediate version, namely a change in the assignment of the Pac-Man to the fields. This means, if the Pac-Man just moves to the neighbour field or to the completely other end of the game field, the difference is the same as atomic diffing techniques are fully agnostic about the spatial aspects of elements. Domain-specific operation detection approaches would be more suitable to represent distances as they are able to represent differences as a set of operations. However, for their computation, atomic diffing is required which is not providing an appropriate guidance to decide if the search goes into the right direction or not. For instance, the Pac-Man could move to any other field and we always get the same fitness value of operations sequences. Therefore, we propose the usage of additional distance metrics to provide additional information on top of difference models in the following section and show their benefits for the search of domain-specific operation-based differences in Section 4.   