
\subsection{Model distance metrics}

Typical model difference tools report metrics on elements added and deleted in terms of instances of metamodel classes, on references changed in terms of instances of metamodel associations, and on attribute value modifications.
As motivated in \Sect{sec:example}, we need metrics that are tailored to the DSL both in terms of the metamodel and the semantics of the transformation.
Therefore, we propose the following three model distance metrics.
These metrics are measured between two models M1 and M2, where M2 is the result of applying a sequence of rules on M1\es{do we need to state that?}. \mw{not necessarily - I would remove it}

The semantics of many modeling formalisms relies on movements of model elements.
Some of their elements are \emph{movable} while others represent \emph{positions} where an element can move to.
Pacman moving on the grid in our example, attributes moving to superclasses in class diagrams, or tokens moving between places in a Petri net are some of many examples where it happens.
Furthermore, some elements are \emph{modifiable} meaning that the transformation changes some of their attribute values, like the score in the rule where Pacman eats food.

Formally, we represent a model as a labeled, attributed multi-graph $G=\langle V,E,l,a \rangle$.
We identify three subsets of nodes $Mov,Pos,Mod \subseteq V$ corresponding to the movable, position, and modifiable objects in the model.
In our running example, grid nodes are the position nodes while Pacman and ghosts are movable nodes.
Note that, in general, $Pos$ may be different for each $v \in Mov$.
Additionally, all three subsets need not to be disjoint nor complete: an object can move between positions, but it can also serve as a position for other movable objects, and it can have attributes modified.
Among the set of edges $e:V \rightarrow V \in E$, we identify two subsets $N,P \subseteq E$.
Neighbor edges $n: Pos \rightarrow Pos \in N$ only connect position nodes, \eg \texttt{left}, \texttt{right}, \texttt{up}, and \texttt{down} references between grid nodes.
Position edges $p: Mov \rightarrow Pos \in P$, like the \texttt{on} references, connect a movable node to a position node.
The label function $l:V \rightarrow \Sigma^*$ assigns a unique string label to each node.
The attribute function $a:V \times \Sigma^* \rightarrow \mathbb{N}$ assigns numerical values to each attribute name of a node.
%$w:Pos \rightarrow \mathbb{N}$ assigns numerical weight to each position edge.

\subsubsection{Move distance}
The move distance of a movable object is the length of the shortest path from its position in model M1 to its position in model M2.
We define $\delta_M(v_1,v_2)$ as the length of the shortest sequence of neighbor edges connecting $v_1$ to $v_2 \in Pos$.
In the M2 case in \Fig{fig:example}, $\delta_M(pacman,grid_{13})=3$ and $\delta_M(ghost,grid_{21})=2$, which is equivalent to the Manhattan distance in this grid layout.
To compute the move distance between M1 and M2, we identify the common connected subgraph $G_{12}$ of their respective graphs $G_1$ and $G_2$.
We define the move distance between two models as:
\[
\Delta_M(G_1,G_2)=\sum_{v_1,v_2 \in Mov_{12}}{\delta_M(p(v_1),p(v_2))} \mbox{, where } l(v_1)=l(v_2)
\]
Here, $p(v_1)$ is the position of $v_1$ in $G_1$ and $p(v_2)$ is its corresponding position in $G_2$.
In our example, $\Delta_M(M1,M2)=5$ and $\Delta_M(M1,M3)=2$.
We rely on the popular Floyd-Warshall algorithm~\cite{Floyd1962,Warshall1962} to compute the shortest path between two nodes in a connected graph.
The dynamic programming implementation takes $O(|Pos|^3)$ to compute all distances between any two nodes and $O(|N|)$ to output the path.


\subsubsection{Element distance}
This metric is concerned with the presence and absence of metamodel class instances between M1 and M2.
This is similar to what a model difference algorithm outputs.
We define the element distance as:
\[
\Delta_E(G_1,G_2)=\frac{|\{v \in V_1 | \nexists v_2 \in V_2, l(v_2)=l(v)\}|+|\{v \in V_2 | \nexists v_1 \in V_1, l(v_1)=l(v)\}|}{|V_1|+|V_2|}
\]
The numerator counts the number of nodes exclusively in each graph.
To normalize the distance as a ratio between 0 and 1, we divide by the total number of nodes in both graphs.
In our example, $\Delta_E(M1,M2)=\frac{3+0}{13+10}=0.13$ and $\Delta_E(M1,M3)=0.13$.
We can interpret this distance as the ratio of objects added or removed between the two models.
In this particular case, 13\% of the objects in M1 have been removed in M2 and in M3.
Note that the element distance is not concerned with edges since they are already taken care of by the move distance.


\subsubsection{Value distance}
The third metric is concerned with the difference in attribute values between objects in M1 and M2.
We assume that any attribute value can be encoded as a unique number, a common practice for metrics~\cite{Bertoa2018}.
We define the value distance of attribute $x$ of node $v$ between $G_1$ and $G_2$ as:
\[
\delta_V(v,x)= \left\{
    \begin{array}{lr}
        |a(v_1,x)| & \mbox{if } a(v,x)=0   \\
        |a(v,x)-a(v_1,x)|/a(v,x) & \mbox{otherwise}
    \end{array}
    \right.
\]
where $v_1 \in Mod_1, v \in Mod_2 \mbox{ and } l(v)=l(v1)$.
Here, we only consider attributes of objects present in both M1 and M2 because the element distance already takes care of the absence and presence of elements.
This distance computes the margin of error needed to obtain the value of $x$ in $v$ from its value in $v_1$.
Note that if $a(v,x)=0$ we replace the denominator by 1.
We define the value distance $\Delta_V(G1,G2)$ between two models as the average of $\delta_V$ for all attributes of all nodes in $G_{12}$.
This calculates the average margin of error between all attribute values of the two models.
In our example, $\Delta_V(M1,M2)=\delta_V(score,value)=\frac{|4-1|}{4}=0.75$ and $\Delta_V(M1,M3)=0.86$.

Typically, if a model difference tool reports the same changes in M2 and M3, the element distance will be the same for both cases as well.
However, the move and value distance will typically discriminate the two as we have seen in the Pacman example.
Furthermore, distance metrics provide a quantitative appreciation of the difference in terms of \emph{``how far''} (thus, comparison distance) M1 is from M2.


\subsection{Adapting distance metrics to the DSL}\label{sec:adapt}

The distance metrics presented in \Sect{sec:metrics} are generic model distances to compare two models.
We now describe how to adapt these metrics for a particular DSL and its semantics.
We aim to produce a distance calculator given the metamodel of the DSL and a set of inplace model transformation rules encoding its semantics.
Typically, these rules have a precondition and a postcondition pattern.

We need to identify the metamodel classes corresponding to the sets of nodes $Pos$ and $Mov$, and the associations corresponding to the sets of edges $N$ and $P$.
The potential candidates for $Mov$ are classes that have an association to another class in the metamodel with cardinality at most 1.
We denote $A$ the potentially movable class and $r$ its association to the other class $B$.
Instances of $A,r,$ and $B$ must be in the precondition of a rule and $r$ must be modified in the postcondition to reference another class instance.
Then, potentially $A$ is a class of movable nodes, $r$ is a position edge type, and $B$ is a class of position nodes.
In our example, these are, among others, the \texttt{Pacman} class, the \texttt{on} association, and the \texttt{GridNode} class, respectively.
It is also possible that $r$ is an association from $B$ to $A$.
Furthermore, it may be that the second instance $A$ refers to is of another type than $B$, say $C$.
If there is a reference $s$ between $B$ and $C$, then $s$ is likely to be a neighbor edge.
Note that this is a necessary condition but not sufficient.
For example, it may be the case where the movable and position classes are connected directly but through an intermediate class.

Similarly, we analyze the classes of the metamodel such that one of its attribute value is modified in the postcondition of a rule.
Such classes define the type of the nodes in $Mod$.

The three distance metrics rely on the label function $l$ to correspond similar nodes between the two models.
For example in \Fig{fig:example}, the grid nodes are identified by their identifier (\eg 11, 12, \ldots).
However, not all classes in the metamodel of the DSL have an identifying attribute.
Since the label function must uniquely identify each node, we must compute a label for each object that does not have one.
We can compute the label structurally.
For example, we can ascertain that there is at most one food on a grid node.
Then the label of a food object can rely on the label of the grid node it is on.
Another case is if we can ascertain that a class is a singleton, then we assign the same label to its instance.
