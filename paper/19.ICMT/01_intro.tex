
The emergence of model-driven engineering (MDE)~\cite{Schmidt2006a} has increased the need for dedicated techniques for model management~\cite{Kolovos2013a}.
In particular, a lot of research was invested in the last decade to develop differencing methods to identify the changes performed between two model versions.
As surveyed in~\cite{StephanC13}, most algorithms aim at computing differences and representing them in the form of difference models which capture the changes between model versions.
Difference models are critical in MDE, being used for various model management tasks, such as metamodel/model co-evolution, versioning or synchronization~\cite{Ruscio2012,Demuth2016}.

While most work has been focusing on differences, less attention was paid to quantifying the distance between model versions.
Distances are useful in addition to differences for several reasons.
First, while different versions of a model may have the same amount of differences, their distance to the base model may be drastically different.
Second, distances can be an additional metric to consider the evolution paths of models to reach a certain setting.
the movement of attributes between different classes in a class diagram.
For example, suppose we have classes $A,B,C,$ and $D$ connected in sequence with associations.
If we move an attribute from $A$ to any other class, we always get the same difference: the attribute deleted from $A$ and added to one of the other classes. However, a distance metric could tell us \textit{``how far''} we have moved the attribute away from $A$, leading to a different distance measures depending in which class it has moved to.

In this work, we present the notion of distance metrics for models as an additional measurement of the difference between models.
Furthermore, we provide a method to derive distance metrics tailored to the domain-specific language (DSL) at hand.
We implemented a software library to automatically generate domain-specific model distance calculators, given the metamodel and the semantical change operators of the DSL.
We apply the distance metrics on the use case of searching for the explanation of model evolution in terms of domain-specific change operations.
Our results show that using distance metrics outperforms common difference models techniques.
%Finally, we discuss potential benefits and challenges of using domain-specific distance measures in comparison to model differences in searching for model evolution explanations in terms of domain-specific change operations. Specifically, the results of running different experiments show...

In Section 2, we overview the background of our approach and motivate our work with a running example for our use case.
In Section 3 we present how to compute the model distance metrics and how to derive them for a particular DSL.
In Section 4, we briefly outline our implementation and the use case for the following evaluation section.
In Section 5, we evaluate the application of these metrics on our use case.
We discuss related work in Section 6 and conclude in Section 7. 