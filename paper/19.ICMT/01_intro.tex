With the emergence of model-driven engineering~\cite{}, the need for dedicated techniques for model management arose. In particular, a lot of research was invested in the last decade to develop differencing methods for models in order to identify the changes performed between two model versions. Several algorithms for computing differences as well as representations for differences have been proposed~\cite{}. Difference models which capture the changes between model versions have been used in different cases such as metamodel/model co-evolution, versioning or synchronization~\cite{}.
Thus, difference models represent an important artefact kind in model-driven engineering.

While differences have been in the focus, less attention was paid to distance computations of model versions. Distances may be useful in addition to differences because of several reasons. First, while different versions of a model may have a similar or even same amount of differences, the distance to a base model may be drastically different. Second, distances may be an additional metric to consider the evolution paths of models to reach a certain setting. As an example think about the movement of attributes between different classes in a class diagram. If we have classes \emph{A}, \emph{B}, \emph{C}, \emph{D} all connected in sequence with associations, and we move an attribute from \emph{A} to another class, we may always have the same difference: the attribute has been deleted in class \emph{A} and has been added to one of the other classes. However, a distance metric could tell us how far we have moved the attribute by getting a different distance measure if the attribute ends up in class \emph{B}, \emph{C}, or \emph{D}.

In order to provide additional measures for understanding how much a model has changed, we present in this paper distance metrics for models which should provide additional information going beyond current difference model types. In addition, we present a method to automatically generate tool support for computing domain-specific distance measures from an annotated version of the language definitions, i.e., metamodels. We provide tool support for our method for the Eclipse Modeling Framework (EMF). Finally, we discuss potential benefits and challenges of using domain-specific distance measures in comparison to model differences in searching for model evolution explanations in terms of domain-specific change operations. Specifically, the results of running different experiments show...

This paper is organized as follows. In Section 2, we provide an overview on the background of our approach and discuss the present gap between differences and distances. Our approach for generating tool support for computing domain-specific distance measures from metamodels is presented in Section 3. In Section 4, we introduce the use case for our evaluation performed in Section 5. In particular, we evaluate the effectiveness of domain-specific distance measures for the case of computing model evolution explanations in terms of domain-specific change operations. In Section 5, we survey related work, and in Section 6, we conclude with a short summary and an outlook on future work. 